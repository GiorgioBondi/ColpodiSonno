\documentclass[a4paper]{article} 
\usepackage[T1]{fontenc} 
\usepackage[utf8]{inputenc} 
\usepackage[italian]{babel}
\begin{document}
\title{Eye tracking per il colpo di sonno}
\author{Giorgio Bondì, Marta Brunetti}
\maketitle
\begin{abstract} Il nostro progetto mira a risolvere il problema del colpo di sonno attarverso un device in grado di riconoscere i sintomi più evidenti come la testa pesante attravero un accelerometro
\end{abstract}
\section*{Il problema}
Ogni anno si verficano 15000 icidenti d'auto dovuti al colpo di sonno e ogni anno muoiono 1000 persone. 
Il colpo di sonno presenta però sintomi estremamente evidenti come palpebre pesanti, lunghi sbadigli e testa pesante.
Troppe volte, però, trascuriamo questi  sintomi, procurandoci così troppi rischi.
\section*{La soluzione}
\subsection*{A cosa abbiamo pensato}
La prima soluzione a cui abbiamo pensato prevedeva la misura della durata di un battito di ciglia prolungato a causa del senso di stanchezza. Il nostro obiettivo iniziale era quello di sfruttare una videocamera per rilevare la chiusura prolungata delle palpebre. 
\subsection*{Primi problemi}
Un battito di ciglia dura circa 0,3s con un intervallo tra i battiti che va dai 2 ai 10 secondi. Il FrameRate della videocamera Logitech a disposizione con le board non era sufficientemente alto per i nostri scopi. Un problema secondario, ma non da sottovalutare, sarebbe stata la gestione di immagini e non di segnali lineari.
\subsection*{La scelta dell'accelerometro}
Non volendo cambiare il fine del nostro progetto, abbiamo pensato, anche grazie all'aiuto degli assistenti, di cambiare l'approccio al problema. Abbiamo deciso di passare all'analisi del movimento del capo, concentrandoci in particolare sul cambiamento di accelerazione del capo durante il fenomeno "Colpo di sonno". 
\subsection*{Linee generali del progetto}
L'idea � quella di rilevare alcuni parametri significativi (accelerazione lungo i tre assi ortonormali, rotazione del capo lungo il piano sagittale) posizionando il nostro sensore sulla nuca di un guidatore durante la simulazione di un colpo di sonno. 
I punti principali del progetto di analisi dei dati sono i seguenti:
\begin{enumerate}
\item acquisizione dati
\item elaborazioni grafiche dei parametri rilevati 
\item filtraggio dei dati
\item calcolo di media sui dati ottenuti e individuazione di possibili outliers o "dati anomali"
\end{enumerate}
\subsection*{Aquisizione dati}
La principale difficolt \'a della acquisizione dei dati � quella di capire le condizioni reali nelle quali il sensore sar� applicato. 
Il sensore durante il tempo di campionamento rileva diversi segnali:
\begin{itemsize}
\item accelerazione del capo lungo gli assi;
\item accelerazione dell'automobile ;
\item accelerazioni casuali come buche o dossi; 
\subsection{end}
Per questi motivi abbiamo deciso di rilevare i dati in due condizioni differenti:
\begin{enumerate}
\item acquisizione dati con il guidatore fermo con soltanto movimento del capo;
\item acquisizione dati con il  guidatore in movimento su una sedia a rotelle e movimento del capo, cercando si simulare il movimento in auto;
\end{enumerate} 
I parametri che abbiamo scelto sono accelerazione sui tre assi e pitch, movimento del capo sul piano sagittale. 

\end{document}
