\documentclass[a4paper]{article} 
\usepackage[T1]{fontenc} 
\usepackage[utf8]{inputenc} 
\usepackage[italian]{babel}
\begin{document}
\title{Eye tracking per il colpo di sonno}
\author{Giorgio Bondì, Marta Brunetti}
\maketitle
\begin{abstract} Il nostro progetto mira a risolvere il problema del colpo di sonno attarverso un device in grado di riconoscere i sintomi più evidenti come la testa pesante attravero un accelerometro
\end{abstract}
\section*{Il problema}
Ogni anno si verficano 15000 icidenti d'auto dovuti al colpo di sonno e ogni anno muoiono 1000 persone. 
Il colpo di sonno presenta però sintomi estremamente evidenti come palpebre pesanti, lunghi sbadigli e testa pesante.
Troppe volte, però, trascuriamo questi  sintomi, prcurandoci così troppi rischi.
\section*{La soluzione}
\subsection*{A cosa abbiamo pensato}
La prima soluzione a cui abbiamo pensato prevedeva la misura della durata di un battito di ciglia prolungato a causa del senso di stanchezza. Il nostro obiettivo iniziale era quello di sfruttare una videocamera per rilevare la chiusura prolungata delle palpebre. 
\subsection*{Primi problemi}
Un battito di ciglia dura circa 0,3s con un intervallo tra i battiti che va dai 2 ai 10 secondi. Il FrameRate dela videocamera Logitech a disposizione con le board non era sufficientemente alto per i nostri scopi. Un problema secondario, ma non da sottovalutare, sarebbe stata la gestione di immagini e non di segnali lineari.
\subsection*{La scelta dell'accelerometro}



\end{document}
